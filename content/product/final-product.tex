\subsection{Producto final}

El producto final se compone de dos sistemas diferentes:

El primero es un servidor de documentos XML en Dublín Core, capaz de responder a las peticiones HTML según el protocolo OAI-PMH. Será capaz de proveer información variada acerca de las publicaciones de los miembros que conforman Morelab en DeustoTech.

Los datos que proporcionará serán:

\begin{itemize}
	\item Información sobre los autores.
	\item Información sobre las tesis.
	\item Información sobre los libros publicados.
	\item Información sobre las revistas.
\end{itemize}

La segunda es una página web desarrollada con tecnologías como HMTL5, CSS3 y JavaScript que hoy en día está en auge y están adquiriendo más y más importancia.

La información será plasmada en la web de modo que cualquier usuario pueda consultarla y filtrarla de forma esquematizada y ordenada.

El objetivo de la aplicación web es poder explotar los datos suministrados por el servidor OAI-PMH, mediante formularios dinámicos, permitiendo realizar tanto búsquedas semánticas avanzadas como facetadas de los recursos facilitados tanto por el servidor de OAI-PMH, como los de SQL y SPARQL.

Añadir que el diseño de la interfaz será resultado de un estudio de los distintas formas de disponer los elementos dentro de un formulario, manteniendo la armonía con los estilos implantados en el sistema actual de LabMan.

Por último destacar, que el diseño de la página web será responsiva, es decir, su diseño se adaptará a distintos tamaños de pantallas como pueden ser las de un ordenador de sobremesa, un portátil, una tablet o un móvil. Además, el funcionamiento de la página deberá ser totalmente intuitiva para que gente con pocos conocimientos de la informática también la pueda usar sin ningún tipo de problema.