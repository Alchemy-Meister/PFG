\chapter*{Resumen}

Deustotech cree que es posible contribuir a un mundo mejor mediante el uso de
las tecnologías de Internet y las Telecomunicaciones.

Como resultado de este pensamiento nació LabMan, un sistema de gestión de grupo
de investigación. Esta aplicación web tiene como objetivo gestionar toda la
información referente a los investigadores, proyectos, publicaciones y tesis de un grupo relacionada entre si. Permite generar diversas gráficas que permiten analizar de forma rápida la evolución y desempeño del equipo de investigación.
Este aplicativo en un claro ejemplo de una web de datos de nueva generación de
portales web, dónde no solo se exportan documentos, sino que habilita la
exportación de datos y APIs, que tienen como propósito facilitar la explotación
de recursos.

Aunque el sistema es capaz de exportar esta información semántica, todavía se ve la necesidad de que este colabore con  sistemas imperantes en la industria para el intercambio de información de recursos académicos y científicos.
Es por ello que se desea dar soporte a OAI, mediante la implementación de su
protocolo OAI-PMH, con el fin de dar servicio a las soluciones del sector que
apostaron en su día por esta tecnología.

Por otra parte, se dispone de dos aplicaciones web que tienen como objetivo
principal la expansión del sistema DMS actual en MORELab. La primera, es capaz
de realizar búsquedas semánticas complejas de los recursos dispuestos por el
servidor de OAI presentado, mientra que la segunda se centra en permitir al
usuario navegar a través las distintas facetas de los mismos.

\vspace{2em}

{\Large\bfseries\sectionfont Descriptores}
\vspace{3\medskipamount}

Biblioteca digital, Buscador Semántico, Aplicación Web, OAI-PMH.