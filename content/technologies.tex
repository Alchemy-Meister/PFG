\chapter{Tecnologías utilizadas}

En esta sección se presentan las diferentes tecnologías que se han utilizado para el desarrollo del proyecto. Las principales tecnologías usado son las siguientes:

\section{Servidor MOAI}

MOAI\cite{MOAI} es una plataforma capaz de recolectar información de diversas fuentes y publicarlas mediante el protocolo \acrshort{oaipmh} desarrollado por Infrae\cite{Infrae} con el fin de satisfacer las necesidades de las instituciones académicas que trabajan con metadatos relacionales y ficheros de información.

Basado en un servidor \acrshort{http} Apache\cite{HTTPApache} e implementado en Python\cite{Python}, MOAI incluye todas las funciones básicas como proveedor de datos del protocolo \acrshort{oaipmh}, indispensables para este proyecto. Por otra parte, permite tanto la extracción la información de documentos \acrshort{xml} como recolectarla de diversos servidores de información tales como Fedora Commons\cite{Fedora}, EPrints\cite{EPrints} o DSpace\cite{DSpace} y almacenar la información cosechada en una base de datos SQLite\cite{SQLite} por defecto.

\subsection{Razón de uso}

Se ha escogido MOAI Server frete a las alternativas facilitadas por la misma comunidad de \acrshort{oai} en \url{https://www.openarchives.org/pmh/tools/tools.php} entre las que se cabe destacar a Fedora, EPrints y DSpace por las siguientes razones:

\begin{itemize}
	\item Su accesible documentación para el desarrollo y extensión del servidor.
	\item Capacidad para sobrescribir el esquema de la base de datos SQLite usado para la extracción de datos del servidor Apache por defecto y sustituirlo por la \acrshort{bd} PostgreSQL\cite{PostgreSQL} utilizada actualmente por LabMan.
	\item Implementado en una tecnología conocida, lo que reduce el tiempo de adaptación a la capa de datos.
\end{itemize}

\section{SQLAlchemy}

\section{Django}

\section{JQuery}

\section{Bootstrap-select}

\section{Bootstrap 3 Datepicker}