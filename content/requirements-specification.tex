\chapter{Especificación de requisitos}

\section{Visión general}

En este capítulo se especifican los requisitos que el proyecto debe satisfacer y que definen el funcionamiento de todo el software que compone este proyecto. Para una mejor comprensión de los mismos, se dividen en los siguientes bloques:

\begin{itemize}
	\item \textbf{Especificación de requisitos del servidor \acrshort{oaipmh}:} en esta sección se recogen los requisitos que debe de satisfacer el servidor \acrshort{oaipmh}, encargado de exponer la información recolectada del repositorio de \acrshort{labman} por medio de ficheros \acrshort{xml} en el formato bibliográfico \acrshort{dc}.

	\item \textbf{Especificación de requisitos del módulo de búsqueda avanzada de publicaciones de la aplicación web:} en está sección se recogen los requisitos que debe satisfacer el módulo encargado de realizar las búsquedas avanzadas de las publicaciones de la aplicación web que extenderá la funcionalidad de \acrshort{labman}.

	\item \textbf{Especificación de requisitos del módulo de búsqueda avanzada de proyectos de la aplicación web:} en está sección se recogen los requisitos que debe satisfacer el módulo encargado de realizar las búsquedas avanzadas de los proyectos de la aplicación web que extenderá la funcionalidad de \acrshort{labman}.
\end{itemize}

\section{Especificación de requisitos del servidor OAI-PMH}

Los requisitos funcionales del servidor de \acrshort{oaipmh} son:

\begin{itemize}
	\item \textbf{RF.0.1}

	\item \textbf{RF.0.2}

	\item \textbf{RF.0.3}

	\item \textbf{RF.0.4}

	\item \textbf{RF.0.5}	
\end{itemize}

Los requisitos no funcionales del servidor de \acrshort{oaipmh} son:

\begin{itemize}
	\item \textbf{RNF.0.1}

	\item \textbf{RNF.0.2}

	\item \textbf{RNF.0.3}

	\item \textbf{RNF.0.4}

	\item \textbf{RNF.0.5}	
\end{itemize}

\section{Especificación de requisitos para la búsqueda avanzada de publicaciones de la aplicación web}

Los requisitos funcionales del módulo de búsquedas avanzadas de publicaciones de la aplicación web servidor son:

\begin{itemize}
	\item \textbf{RF.0.1}

	\item \textbf{RF.0.2}

	\item \textbf{RF.0.3}

	\item \textbf{RF.0.4}

	\item \textbf{RF.0.5}	
\end{itemize}

Los requisitos no funcionales del módulo de búsquedas avanzadas de publicaciones de la aplicación web servidor son:

\begin{itemize}
	\item \textbf{RNF.0.1}

	\item \textbf{RNF.0.2}

	\item \textbf{RNF.0.3}

	\item \textbf{RNF.0.4}

	\item \textbf{RNF.0.5}	
\end{itemize}


\section{Especificación de requisitos para la búsqueda avanzada de proyectos de la aplicación web}

Los requisitos funcionales del módulo de búsquedas avanzadas de proyectos de la aplicación web servidor son:

\begin{itemize}
	\item \textbf{RF.0.1}

	\item \textbf{RF.0.2}

	\item \textbf{RF.0.3}

	\item \textbf{RF.0.4}

	\item \textbf{RF.0.5}	
\end{itemize}

Los requisitos no funcionales del módulo de búsquedas avanzadas de proyectos de la aplicación web servidor son:

\begin{itemize}
	\item \textbf{RNF.0.1}

	\item \textbf{RNF.0.2}

	\item \textbf{RNF.0.3}

	\item \textbf{RNF.0.4}

	\item \textbf{RNF.0.5}	
\end{itemize}


\section{Criterios de validación}

Los requisitos previamente expuestos en las secciones anteriores están sujetos a procesos de validación antes de la entrega final del proyecto.Con el objetivo de comprobar el cumplimiento de los requisitos, las prestaciones del producto final serán contrastadas con estos requisitos iniciales, siendo posible su modificación para satisfacer las necesidades del cliente estudiando en cada caso las implicaciones que esto generaría en los plazos de entrega y el presupuesto. El método de desarrollo está guiado por pruebas, de forma que el cumplimiento exitoso de dichas pruebas validará los distintos requisitos del sistema objetivamente, mientras que si dichas pruebas no se llegarán a ejecutar satisfactoriamente implicaría la existencia de requisitos incompletos en el sistema.

El grado de cumplimiento del proyecto estará directamente relacionado con el porcentaje de requisitos no cumplidos, siendo este usado para evaluar el grado de completitud objetivamente.

Del mismo modo, cualquier implementación que mejore la estabilidad o funcionalidad del sistema que no esté reflejado en los requisitos iniciales se considerará una parte extra de la evaluación del proyecto por el director del mismo.