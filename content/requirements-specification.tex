\chapter{Especificación de requisitos}

\section{Visión general}

En este capítulo se especifican los requisitos que el proyecto debe satisfacer y que definen el funcionamiento de todo el software que compone este proyecto. Para una mejor comprensión de los mismos, se dividen en los siguientes bloques:

\begin{itemize}
	\item \textbf{Especificación de requisitos del servidor \acrshort{oaipmh}:} en esta sección se recogen los requisitos que debe de satisfacer el servidor \acrshort{oaipmh}, encargado de exponer la información recolectada del repositorio de \acrshort{labman} por medio de ficheros \acrshort{xml} en el formato bibliográfico \acrshort{dc}.

	\item \textbf{Especificación de requisitos del módulo de búsqueda avanzada de publicaciones de la aplicación web:} en está sección se recogen los requisitos que debe satisfacer el módulo encargado de realizar las búsquedas avanzadas de las publicaciones de la aplicación web que extenderá la funcionalidad de \acrshort{labman}.

	\item \textbf{Especificación de requisitos del módulo de búsqueda avanzada de proyectos de la aplicación web:} en está sección se recogen los requisitos que debe satisfacer el módulo encargado de realizar las búsquedas avanzadas de los proyectos de la aplicación web que extenderá la funcionalidad de \acrshort{labman}.
\end{itemize}

\section{Especificación de requisitos del servidor OAI-PMH}

Los requisitos funcionales del servidor de \acrshort{oaipmh} son:

\begin{itemize}
	\item \textbf{RF.0.1}

	El servidor debe transformar datos almacenados en una base de datos relacional en \acrshort{xml} respetando la especificación del esquema bibliográfico \acrshort{dc}.

	\item \textbf{RF.0.2}

	El servidor debe visualizar la información como \acrshort{xml} en texto plano en navegadores web convencionales.

	\item \textbf{RF.0.3}

	El servidor tiene que generar los \textit{sets} sobre cada uno de los tipos de publicaciones para facilitar la recuperación de metadatos selectiva.	
\end{itemize}


\section{Especificación de requisitos para la búsqueda avanzada de publicaciones de la aplicación web}

Los requisitos funcionales del módulo de búsquedas avanzadas de publicaciones de la aplicación web son:

\begin{itemize}
	\item \textbf{RF.0.1}

	El formulario extendido debe estar oculto inicialmente para permitir realizar consultas sencillas rápidamente.

	\item \textbf{RF.0.2}

	La aplicación debe ser capaz de mostrar las publicaciones resultantes de la consulta solicitada en la búsqueda avanzada.

	\item \textbf{RF.0.3}

	Debe permitir realizar búsquedas por rango entre dos fechas o establecer una fecha específica.
	
\end{itemize}

Los requisitos no funcionales del módulo de búsquedas avanzadas de publicaciones de la aplicación web son:

\begin{itemize}
	\item \textbf{RNF.0.1}

	La aplicación tiene que mantener el \textit{Look and Feel} originario de \acrshort{labman}, por lo que su diseño y elementos básicos deberán hacer uso o basarse en los componentes facilitados por Bootstrap 3\cite{Bootstrap}.

	\item \textbf{RNF.0.2}

	Los elementos del buscador avanzado han de ser responsiva para ser adaptable en diferentes dispositivos, como ordenadores de sobremesa, tabletas y móviles.

	\item \textbf{RNF.0.3}

	Las distintas secciones en la que se dispone el buscador de publicaciones tienen que tener un diseño intuitivo para facilitar su uso.	
\end{itemize}


\section{Especificación de requisitos para la búsqueda avanzada de proyectos de la aplicación web}

Los requisitos funcionales del módulo de búsquedas avanzadas de proyectos de la aplicación web son:

\begin{itemize}
	\item \textbf{RF.0.1}

	El formulario extendido debe estar oculto inicialmente para permitir realizar consultas sencillas rápidamente.

	\item \textbf{RF.0.2}

	La aplicación debe ser capaz de mostrar las proyectos resultantes de la consulta solicitada en la búsqueda avanzada.

	\item \textbf{RF.0.3}

	Debe permitir realizar búsquedas por rango entre dos fechas o establecer una fecha específica.
	
\end{itemize}

Los requisitos no funcionales del módulo de búsquedas avanzadas de proyectos de la aplicación web son:

\begin{itemize}
	\item \textbf{RNF.0.1}

	La aplicación tiene que mantener el \textit{Look and Feel} originario de \acrshort{labman}, por lo que su diseño y elementos básicos deberán hacer uso o basarse en los componentes facilitados por Bootstrap 3.

	\item \textbf{RNF.0.2}

	Los elementos del buscador avanzado han de ser responsiva para ser adaptable en diferentes dispositivos, como ordenadores de sobremesa, tabletas y móviles.

	\item \textbf{RNF.0.3}

	Las distintas secciones en la que se dispone el buscador de proyectos tienen que tener un diseño intuitivo para facilitar su uso.
\end{itemize}


\section{Criterios de validación}

Los requisitos previamente expuestos en las secciones anteriores están sujetos a procesos de validación antes de la entrega final del proyecto. Con el objetivo de comprobar el cumplimiento de los requisitos, las prestaciones del producto final serán contrastadas con requisitos aquí expuesto, siendo posible su modificación para satisfacer las necesidades del cliente, estudiando en cada caso las implicaciones que esto generaría en los plazos de entrega y el presupuesto. 

El método de desarrollo está guiado por pruebas, de forma que el cumplimiento exitoso de dichas pruebas validará los distintos requisitos del sistema objetivamente. Si dichas pruebas no se llegarán a ejecutar satisfactoriamente, implicaría la existencia de requisitos incompletos en el sistema.

El grado de cumplimiento del proyecto estará directamente relacionado con el porcentaje de requisitos no cumplidos, siendo este usado para evaluar el grado de completitud objetivamente.

Del mismo modo, cualquier implementación que mejore la estabilidad o funcionalidad del sistema que no esté reflejado en los requisitos iniciales se considerará una parte extra de la evaluación del proyecto por el director del mismo.