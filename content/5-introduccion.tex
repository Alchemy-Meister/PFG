\chapter{Introducción}\label{cha:introduccion}

\section{Presentación del Documento}

El presente informe describe el proyecto de desarrollo Gestión de repositorios semánticos compatible con el estándar OAI-PMH, un aplicativo que pretende extender a LabMan, el DMS (Data Management System) de los grupos de Internet y Telecomunicaciones de DeustoTech, detallando tanto los objetivos que se pretenden alcanzar con el proyecto, como las fases, actividades y recursos necesarios para llevarlo a cabo.

El contenido de este documento se estructura en torno a los siguientes productos:

\begin{itemize}
	\item \textbf{Definición de proyecto:}
		
	Establecimiento del objetivo fundamental del proyecto, especificando cuáles son los aspectos funcionales que lo comprenden y cuáles son los que quedan excluidos.
	
	\item \textbf{Producto final:}
		
	Especificación de la solución elegida que va a construir el proyecto en cuestión.
	
	\item \textbf{Descripción de la realización:}

	Realización y definición de las diferentes actividades cuyo desarrollo va a permitir la realización y consecución del objetivo del proyecto.

	\item \textbf{Organización:}
	
	Definición del equipo de trabajo que desarrollará el proyecto, así como su estructura organizativa, sistema de gestión y seguimiento del trabajo.

	\item \textbf{Condiciones de ejecución:}

	Definición del entorno de trabajo, de los criterios sobre los que se van a realizar las sucesivas recepciones, así como el tratamiento que se va a establecer para aquellos casos que puedan ser considerados como modificaciones o mejoras en el planteamiento inicial del proyecto.

	\item \textbf{Planificación:}
	Estimación de cargas y duración de las diferentes actividades del proyecto, así como su asignación a los diferentes miembros del equipo y su planificación en el tiempo.

	\item \textbf{Valoración económica:}
	Determinación del valor correspondiente a este proyecto, de los hitos de facturación y de la forma de pago.
\end{itemize}

\section{Motivación}

DeustoTech cree que es posible contribuir a un mundo mejor mediante el uso de las tecnologías de Internet y Telecomunicaciones.

Como resultado de este pensamiento nació LabMan, un sistema de gestión de grupos de investigación. Esta aplicación web tiene como objetivo gestionar toda la información referente a los investigadores, proyectos y publicaciones de un grupo relacionada entre si. Permite generar diversas gráficas que permiten analizar de forma rápida la evolución y desempeño del grupo de investigación.

Este aplicativo es un claro ejemplo de una web de datos de nueva generación de portales web, dónde no solo se exportan documentos, sino que habilita la exportación datos y APIs, que tienen como propósito facilitar la explotación de datos.

Aunque es capaz de exportar esta información semántica, todavía se ve la necesidad de que el sistema colabore con sistemas imperantes en la industria para el intercambio de información de recursos académicos y científicos.
Es por ello que se desea dar soporte a OAI, mediante la implementación de su protocolo Open Archive Initiative Protocol for Metadata Harvesting, más comúnmente conocido por sus siglas OAI-PMH, para poder comunicarse y dar servicio a soluciones del sector que apostaron por esta tecnología, favoreciendo por otra parte una mayor explotación de la información.

Así mismo se desea fomentar accesibilidad a dichos recursos, teniendo en especial consideración a aquellos usuarios que no disponen de conocimientos informáticos, por medio de un cliente web que sirva tanto de buscador como de filtrador.

Esta accesibilidad se garantizará mediante un estudio exhaustivo de los distintas formas en las que se pueden disponer los formularios y de cómo el usuario interactúa con ellos, teniendo como resultado una interfaz de usuario que asegura una experiencia intuitiva.
