\chapter{Introducción}\label{cha:introduccion}

\section{Presentación del Documento}

El presente informe describe el proyecto de desarrollo Gestión de repositorios semánticos compatible con el estándar \acrshort{oaipmh}, un aplicativo que pretende extender a \acrshort{labman}, el \acrlong{dms} de los grupos de Internet y Telecomunicaciones de DeustoTech, detallando tanto los objetivos que se pretenden alcanzar con el proyecto, como las fases, actividades y recursos necesarios para llevarlo a cabo.

El contenido de este documento se estructura en torno a los siguientes productos:

\begin{itemize}
	\item \textbf{Definición de proyecto:}
		
	Establecimiento del objetivo fundamental del proyecto, especificando cuáles son los aspectos funcionales que lo comprenden y cuáles son los que quedan excluidos.
	
	\item \textbf{Producto final:}
		
	Especificación de la solución elegida que va a construir el proyecto en cuestión.
	
	\item \textbf{Descripción de la realización:}

	Realización y definición de las diferentes actividades cuyo desarrollo va a permitir la realización y consecución del objetivo del proyecto.

	\item \textbf{Organización:}
	
	Definición del equipo de trabajo que desarrollará el proyecto, así como su estructura organizativa, sistema de gestión y seguimiento del trabajo.

	\item \textbf{Condiciones de ejecución:}

	Definición del entorno de trabajo, de los criterios sobre los que se van a realizar las sucesivas recepciones, así como el tratamiento que se va a establecer para aquellos casos que puedan ser considerados como modificaciones o mejoras en el planteamiento inicial del proyecto.

	\item \textbf{Planificación:}
	Estimación de cargas y duración de las diferentes actividades del proyecto, así como su asignación a los diferentes miembros del equipo y su planificación en el tiempo.

	\item \textbf{Valoración económica:}
	Determinación del valor correspondiente a este proyecto, de los hitos de facturación y de la forma de pago.
\end{itemize}

\section{Introducción a LabMan}

Es este apartado se realiza una breve introducción a \acrfull{labman}, el \acrlong{dms} del grupo de investigación MoreLab, así como de las tecnologías que lo conforman. Este contexto facilitará la comprensión de los próximos capítulos, además de describir los fundamentos clave para desarrollo de este proyecto.

\begin{figure}[!htp]
	\centering
	\includegraphics[scale=0.15]{fig/morelab-logo}
	\caption{Logotipo de MORELab}
\end{figure}

Gestionar la información no siempre es una tarea trivial y más aún cuando hay que tratar con los datos que componen varias entidades como pueden ser los proyectos, investigadores, publicaciones, eventos, etc. dentro del ámbito de la investigación. La mayoría de los grupos de investigación utilizan sistemas de gestión de contenido tales como Joomla!\cite{joomla}, WordPress\cite{wordpress} o Drupal\cite{drupal} para exponer sus datos. Sin embargo, para extraer la información de estos \acrshortpl{cms} se requieren herramientas externas para llevar a cabo técnicas de análisis de datos. 
Para hacer uso de estas herramientas normalmente hace falta generar documentos adicionales, tales como \acrshort{csv}, hojas de cálculo, ficheros de texto, generado información redundante, que provoca dificultades a la hora de actualizar los datos y la calidad de los mismos. Esta situación empeora cuando además se disponen de distintas fuentes para la obtención de información, como pueden ser las paginas web personales de los investigadores en los que se muestran sus logros a lo largo de su carrera, la información financiera gestionada por su propio departamento, etc.

\begin{figure}[!htp]
	\centering
	\includegraphics[scale=0.21]{fig/labman-chart}
	\caption{\acrshort{labman}: Gráfico generado por tripletes \acrshort{rdf}}\label{fig:labmanchart}
\end{figure}

Del esfuerzo para gestionar la información grupo de investigación de MoreLab nace \acrshort{labman}, una aplicación desarrollada en Python\cite{Python} por medio del framework para desarrollo web Django\cite{Django} y que sustituye a la antigua solución Joomla! para la publicación de los datos sobre las publicaciones en \acrfull{rdf}\cite{RDF}. Su principal objetivo es gestionar no solo la información relacionada a las publicaciones, sino que va más allá publicando la información relacionada a los proyectos del equipo, sus financiaciones, integrantes de proyecto, noticias de una forma gráfica e intuitiva tal y como se puede ver en la figura \ref{fig:labmanchart}, diferenciadose de otros \acrshortpl{cms} por apostar por la exposición de los datos como \acrlong{lod}\cite{linkeddata}, disponible por todos sin restricción de derechos de \textit{copyrights} o patentes. Si es verdad que existen extensiones para que los \acrshort{cms} puedan publicar los datos almacenados en estos sistemas como \acrshortpl{rdf}, no permiten acceder a estos mediante un punto de salida \acrshort{sparql}, por lo que no se pueden realizar consultas complejas a entidades externas ni aprovechar las ventajas que se producen por publicarlos siguiendo las prácticas de \acrshort{ld}.

Mediante el uso de los datos enlazados es posible identificar cada entidad por medio de una \acrshort{uri}. Además de evitar la redundancia de datos esto permite la creación de relaciones entre distintas instancias, lo que deriva en la posibilidad de descubrir patrones dentro de un set de datos.\cite{pena_visual_2014}
\section{Introducción a SQL}

Como su propio nombre indica, \acrfull{sql} es un lenguaje de programación, estandarizado por \acrshort{iso}\cite{ISO} y \acrshort{ansi}\cite{ANSI}, diseñado para gestionar la información almacenada en los sistemas de gestión de bases de datos relacionales, como por ejemplo PostgreSQL\cite{PostgreSQL}, MySQL\cite{MySQL}, SQLite\cite{SQLite}, por medio de consultas estructuradas en inglés.

Las consultas representan las operaciones más comunes y esenciales del lenguaje \acrshort{sql} para la recopilación de información dentro de una \acrshort{bd}. Estas consultas se realizan por medio de la sentencia ``SELECT'', que pueden ser complementadas por medio de clausulas para realizar búsquedas específicas. De todas las clausulas disponibles hay que destacar la siguientes:

\begin{itemize}
	\item \textbf{FROM:} Indica de que tabla o tablas de debe extraerse la información.
	\item \textbf{WHERE:} Sirve para delimitar las tuplas o filas de la tablas en las que se realiza la búsqueda. Si las filas no cumplen las condiciones expuestas en esta clausula, serán excluidas del resultado.
	\item \textbf{ORDER BY:} Esta es la única forma para definir el criterio de ordenación los resultados en \acrshort{sql}, sin esta clausula el orden sería aleatorio.
\end{itemize}

\lstinputlisting[language=SQL, frame=single, label={lst:selectsql}, caption=Ejemplo de sentencia SELECT en \acrshort{sql}]{content/code/sql/select-example.sql}

En esta consulta \acrshort{sql} (ver algoritmo \ref{lst:selectsql}) se solicitan todos los nombres y las ciudades de los clientes que residan en Suecia.
\section{Introducción a SPARQL}

\acrshort{sparql}\cite{SPARQL_language} es el acrónimo recursivo del inglés \acrlong{sparql} que hace referencia tanto a el lenguaje estandarizado por la \acrshort{rdf} \acrshort{dawg} de la \acrshort{w3c}\cite{W3C} para consultas a grafos \acrshort{rdf} como para el protocolo de invocación de consultas \acrshort{sparql} remotas.

El lenguaje de consultas \acrshort{sparql} (actualmente en la versión 1.1), permite tanto buscar como manipular grafos \acrshort{rdf} disponibles en la web o bases de datos semánticas almacenados como tripletas, en otras palabras, define un lenguaje equivalente a \acrshort{sql} a excepción de que este se utiliza exclusivamente para bases de datos semánticas.

\lstinputlisting[language=SQL, otherkeywords={PREFIX}, frame=single, label={lst:selectsparql}, caption=Ejemplo de sentencia SELECT en \acrshort{sparql}]{content/code/sparql/select-example.sparql}

Al igual que en \acrshort{sql} las consultas en \acrshort{sparql} constituyen las operaciones más comunes y esenciales del lenguaje. Su estructura es es muy similar a la ya vista en el algoritmo \ref{lst:selectsql}, haciendo uso de la misma sentencia ``SELECT''.

Esta consulta \acrshort{sparql} (ver algoritmo \ref{lst:selectsparql}) añade un nuevo elemento ``PREFIX'' a lo anteriormente visto en \acrshort{sql}, que tiene como función almacenar \acrshortpl{uri} para reducir la longitud de de las mismas a la hora de acceder a sus atributos. La consulta en sí busca en los sujetos ?planttype y los objetos ?name que estén relacionado con el predicativo plant:planttype y devuelve solo sus nombres.
\section{Introducción a OAI-PMH}

\acrfull{oaipmh} (actualmente disponible en su versión 2.0) es un protocolo para la transmisión de metadatos en Internet que surgió del esfuerzo de mejorar y abrir el acceso a archivos de publicaciones electrónicas (e-prints) y en general a un gran rango de materiales digitales que ha despertado el interés de la comunidad de bibliotecarios.\cite{JM_OAI}

\begin{figure}[!htp]
	\centering
	\includegraphics[scale=.15]{fig/oai_flow}
	\caption{Diagrama de flujo del servidor \acrshort{oaipmh}}\label{fig:oaiflow}
\end{figure}

El protocolo \acrshort{oaipmh} se basa en una arquitectura cliente-servidor en la que el cliente realiza las consultas por medio de transacciones \acrshort{http} GET o POST constituidas por un conjunto de opciones del tipo clave=valor y se devuelve un documentos \acrshort{xml} bajo al menos el estandar \acrfull{dc} según dicta la especificación de la implementación mínima del protocolo \acrshort{oaipmh} para los proveedores de información.\cite{OAIPMH_implementers}

Como se ve en la figura \ref{fig:oaiflow}\cite{oai_implementation} el cliente puede realizar seis peticiones al servidor en función del valor especificado en el `verb' de la consulta, las peticiones pueden ser:

\begin{itemize}
	\item \textbf{Idenfity:} Envía información sobre el servidor así como el nombre del repositorio, \acrshort{url} base, correo electrónico del administrador del servidor, etc.
	\item \textbf{ListMetadataFormats:} Envía la lista de formatos en los que se encuentran disponibles los metadatos que al menos deben estar disponibles en \acrshort{dc}.
	\item \textbf{ListSets:} Envía una lista con los términos opcionales creados por el servidor para facilitar la recuperación de metadatos selectiva de los registros, posibilitando que un cliente pueda solicitar los registros pertenecientes a una clase en concreto. Estos términos representan una clasificación de los recursos según varias entradas. Los sets pueden estar compuestos por listas simple o formar una jerarquía.
	\item \textbf{ListIdentifiers: } Devuelve la cabecera de hasta un máximo de 100 registros por petición. Los identificadores pueden filtrarse por un rango entre dos fechas de creación o modificación o por un distintos `Sets' definidos por el servidor.
	\item \textbf{ListRecords: } Realiza la misma tarea que la petición \textit{ListIdentifiers} a excepción de devolver el registro completo con sus metadatos, en lugar de incluir solo la cabecera del recurso. En caso de que la petición resulte una lista de más de 100 recursos, al igual que en la petición \textit{ListIdentifiers}, al final de la lista \acrshort{xml} se devolverá una clave `resumptionToken' que deberá ser utilizada por el cliente para continuar la devolución de los siguientes 100 registros en otra petición \acrshort{http} independiente.
	\item \textbf{GetRecord: } Petición utilizada para devolver un registro en concreto (véase algoritmo \ref{lst:oaipmhgetrecord}, siendo necesario especificar el identificador del recurso solicitado y el formato bibliográfico en el que se desea que se devuelva.
\end{itemize}

\lstinputlisting[language=XML, frame=single, label={lst:oaipmhgetrecord}, caption=Ejemplo de petición GetRecord de \acrshort{oaipmh}]{content/code/xml/get-record-example.xml}

El \acrshort{xml} generado por el servidor presenta la siguiente estructura para la extracción de los recursos:

\begin{itemize}
	\item \textbf{Información sobre la transacción:} El comienzo del \acrshort{xml} siempre está compuesto por atributos smlns, smlns:xsi y xsi:schemaLocation que sirven para definir `namespace' y el esquema de \acrshort{oaipmh} del \acrshort{xml}.

\end{itemize}
\section{Fortalezas y debilidades de las tecnologías}

Habiendo introducido anteriormente las tecnologías que dispone \acrshort{labman} para su exposición a través del portal web y/o accesible desde las \acrshortpl{api} que proporciona, es necesario realizar un estudio sobre las ventajas y desventajas de la nueva tecnología introducida al sistema respecto a las ya presentes y valorar como afectan al sistema.

\subsection{OAI-PMH}

\subsubsection{Fortalezas}

\begin{itemize}
	\item El protocolo se basa en realizar peticiones \acrshort{http} por lo que no hay que configurar ningún puerto en concreto para hacerlo funcionar.

	\item Al ser un protocolo arraigado en la industria, publicaciones académicas y en las comunidades científicas, convirtiéndose en gigantes ``bibliotecas digitales'', hace que este medio sea perfecto para indexar y explotar grandes volúmenes de información actualizada publicada por estos repositorios.

	\item Las peticiones son sencillas de realizar y no requiere conocimiento alguno de ningún lenguaje de programación. Tan solo de saber estructurar un petición GET o POST con uno de los seis verbos que establece el protocolo \acrshort{oaipmh}, así como definir el formato de los metadatos en los que se desea que se genere la respuesta.
\end{itemize}

\subsubsection{Debilidades}

\begin{itemize}
	\item Al ser un protocolo que gestiona vastas cantidades de información, la gran demanda de los proveedores de servicios podría provocar que las peticiones de estos se conviertan en un ataque de denegación de servicios. Lo que hace que factores como el balanceo de carga, el control de flujo y las redirecciones a servidores secundarios o de ``backup'' sea  esencial por parte de los repositorios. Así mismo es sumamente importante que los clientes no bombardeen a el servidor en caso de encontrarse en situaciones de error y se aconseja que se sigan buenas prácticas a la hora de configurar a los ``robots'' que realizan estas en segundo plano.

	\item El protocolo carece de filtros avanzados para realizar búsquedas. Al ser un protocolo que promueve la preservación de datos digitales, no ha sido diseñado un sistema de búsquedas o filtros avanzados, tan solo permite filtrar por los “Sets” definidos por el propio repositorio, por fecha inicio y fecha de fin buscando por un identificador en concreto. La responsabilidad de implementar un sistema de búsquedas recae por tanto a el proveedor de servicios y su extracción dependerá del sistema de almacenamiento de la información utilizado por el cliente de \acrshort{oaipmh}.

	\item El rendimiento es dependiente de la implementación del servidor. El proveedor de datos puede recopilar la información de múltiples fuentes, pudiendo optar cualquier tipo de sistema de almacenamiento como ficheros \acrshort{xml}, texto plano, NoSQL\cite{NoSQL}, \acrlongpl{bd} relacionales, entre muchos otros. Esto implica que el rendimiento estará sujeto al tiempo de acceso, filtrado de la información y al tiempo de generación de la respuesta \acrshort{xml} del servidor \acrshort{oaipmh}.
\end{itemize}

\subsection{SQL}

\subsubsection{Fortalezas}

\begin{itemize}
	\item Alto rendimiento, las peticiones \acrshort{sql} pueden ser usadas para acceder a grandes volúmenes de información de la base de datos de manera rápida y eficiente.

	\item El lenguaje ha sido estandarizado por \acrshort{ansi} y por \acrshort{iso}, lo que hace posible que se pueda reutilizar el código \acrshort{sql} en distintas bases de datos teniendo que realizar modificaciones mínimas.

	\item Permite el almacenamiento de objetos en la base de datos, dado que los \acrshort{dbms} orientados a objetos\cite{OODB} son una extensión de las bases de datos relacionales.

	\item Es capaz de realizar consultas complejas, buscando por cada campo definido en las tablas, realizar filtros avanzados con \textit{inner queries}, ejecutar todo funciones y \textit{joins} con otras tablas, lo que lo convierte en una herramienta idónea para la recolección de datos.
\end{itemize}

\subsubsection{Debilidades}

\begin{itemize}
	\item Aunque las bases de datos \acrshort{sql} se conforman a los estándares \acrshort{ansi} e \acrshort{iso} muchas de las bases de datos \acrshort{sql} implementan algunas de sus funcionalidades como extensiones propietarias para asegurar la permanencia de sus usuarios.

	\item Realizar consultas a una tabla es meramente intuitivo, pero a medida que se añaden condiciones, filtros, funciones, \textit{inner queries} y \textit{joins} con otras tablas la dificultad aumenta, haciendo el lenguaje difícil, para consultas avanzadas.
\end{itemize}

\subsection{SPARQL}

\subsubsection{Fortalezas}

\begin{itemize}
	\item Alto soporte a consultas semiestructuradas, como por ejemplo a datos con estructuras impredecibles o poco fiables.

	\item \acrshort{sparql} realiza consultas a grafos \acrshortpl{rdf} y estos están compuestos por varias tripletas que expresan relaciones binarias entre los recursos. Esta característica permite a las consultas \acrshort{sparql} realizar \textit{joins} implícitas entre los distintos recursos, convirtiendolo en un lenguaje apropiado para realizar búsquedas en fuentes de información dispares en una sola consulta.

	\item \acrshort{sparql} ha sido diseñado para soportar consultas en un entorno web, en los que los nombres de los grafos se identifican mediante URIs, por otra parte es común que las implementaciones de \acrshort{sparql} recojan la información de los grafos por medio peticiones \acrshort{http} GET sobre las \acrshortpl{uri} de los grafos.

	\item Existen herramientas que permiten el uso de \acrshort{sparql} para consultar contenido que no está almacenado en RDF, tales como \acrshort{ldap}, \acrshort{xml}, \acrshort{sql}, entre otros. Entre estas herramientas caben destacar el servidor D2R\cite{D2R_Server}, el Adaptador Jena para bases de datos Oracle\cite{JENA_ORACLE}, SquirrelRDF\cite{SquirrelRDF}, etc.\cite{MAPING_SPARQL} que ofrecen servicios de mapeo de \acrfull{s2s}.

	\item Así mismo la palabra clave de un grafo permite adquirir datos del lugar de origen de dicha información. Los grafos pueden ser usados para descubrir las \acrshortpl{uri} de grafos que contengan los datos que coincidan con dicha consulta en cuestión.
\end{itemize}

\subsubsection{Debilidades}

\begin{itemize}
	\item Por lo general, las consultas en \acrshort{sparql}, de no usar estrategias el cacheo de consultas\cite{SPARQL_Performance}, son muy lentas.

	\item Al ser una tecnología joven, no son muchos los que hacen uso de esa tecnología, es por eso que aún no haya muchos repositorios de datos que puedan recolectarse por medio de consultas \acrshort{sparql} comparado al basto despliegue de otras tecnologías veteranas como \acrshort{sql} o \acrshort{xpath}\cite{XPath}.

	\item No da soporte a las consultas jerárquicas o transitivas. \acrshort{sparql} no facilita realizar consultas que presenten relaciones transitivas o estructuras jerárquicas dentro de un grafo, algo que por el contrario si soporta \acrshort{xquery} mediante los \textit{Axis}, capaz de relacionar un nodo en cuestión, con su padre y sus hijos\cite{XQueryAxes}.

	\item Es una tecnología inmadura en la que se hace notar la ausencia de procesamiento explícito como el de \acrshort{xquery} o la optimización del ya veterano \acrshort{sql}.
\end{itemize}

\sebsection{Conclusiones}

