\subsection{Alcance del proyecto}

Atendiendo a las premisas señaladas anteriormente, las funcionalidades que deberá soportar este proyecto serán:

\begin{itemize}
	\item Un servidor capaz de conectarse al repositorio de LabMan y extraer la información actualizada, en forma de metadatos, sobre las publicaciones y sus autores correspondientes de acuerdo con el estándar Dublin Core, respondiendo a las peticiones \acrshort{http} de acuerdo con el protocolo \acrshort{oaipmh}.
	
	\item Un cliente web capaz de realizar consultas semánticas y facetadas complejas y presentarlas intuitivamente a los usuarios, compuesto por formularios que se dispondrán de forma sencilla en primera instancia, para dar la posibilidad de realizar consultas rápidas y simples sin abrumar a los usuarios por la longitud del mismo. Pero, a su vez, han de permitir dar la posibilidad de expandir los campos con el fin de introducir datos más específicos para realizar consultas más elaboradas.

	\item La plataforma estará diseñada de una manera intuitiva, para que así, personas con pocos conocimientos de la informática también la puedan usar sin ningún tipo de problema. Además ha de ser responsiva, es decir, su diseño se adaptará a distintos tamaños de pantallas como pueden ser las de un ordenador de sobremesa, un portátil, una tableta o un móvil, donde poder plasmar toda la información de una manera legible para los humanos.
\end{itemize}

El proyecto se centrará solamente en el repositorio de Labman, a las tablas relacionadas a las publicaciones (Autores, Tesis, Libros, Tags, etc.), la incorporación de los demás repositorios de DeustoTech quedan aplazados para futuras revisiones del proyecto una vez concluido este.