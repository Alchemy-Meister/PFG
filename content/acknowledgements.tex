\chapter*{Agradecimientos}
\addcontentsline{toc}{chapter}{Agradecimientos}

No podríamos dar finalizada la memoria del presente proyecto, sin agradecer a todos aquellos que me han apoyado u ofrecido su ayuda durante estos meses de arduo trabajo.
En especial quisiera dar las gracias a: 

\begin{itemize}
	\item \textbf{Mi padre, Abilio}, que ha sacrificado 46 años de su juventud trabajando para poder ofrecerme la posibilidad realizar estos estudios académicos. Pero además me ha apoyado y dado ánimos en los momentos más duros, en los que de no haber sino por él hubiera tirado la toalla.

	\item \textbf{Mi madre, Begoña}, que aunque muchas veces no la haya valorado siempre ha estado ahí preocupandose por mí, haciendome la vida más cómoda de un modo u otro. De no haber sido por su insistencia, y de todas esas horas que ha pasado cerca de mi apoyandome en lo que ha podido, dejando sus tareas de lado, no hubiera podido llegar a donde estoy.

	\item \textbf{Aritz Bilbao Jayo}, compañero y amigo de gran corazón, que desde que lo conocí en primero de carrera no ha dejado de sorprenderme de sus capacidades y de lo tremendamente humano que es. Se ha preocupado por mí y me ha ayudado durante toda la carrera resolviendome las dudas que me hayan podido surgir, que cada vez van en aumento lamentablemente. 
	Además fue él quien me recomendó para trabajar en MORElab y quien que me introdujo en el desarrollo web, sin él no hubiera podido avanzar tan deprisa.
	
	Espero seguir pasando buenos momentos junto a él durante venideras ediciones de la Euskal encounter. Sin duda se merece una TARDIS.

	\item \textbf{Iban Eguia Moraza}, preciado amigo que me ha apoyado en tiempos difíciles de la carrera. Es además un sorprendente hombre con una capacidad por encima de la media que me ha enseñado las virtudes del \textit{Open-source}. Su último gran consejo, a la hora de redactar estas lineas, fue el convencerme de utilizar \hologo{LaTeX}, que de muchas horas de quebraderos de cabeza me ha librado.
	
	Más importante aún, me ha enseñado a valorar a los que me rodean y a ser más humano y sociable, cualidades que aún tengo que esforzarme en potenciar.

	\item \textbf{Diego López-de-Ipiña González-de-Artaza}, No solo ha sido el director de este proyecto, sino que además ha sido mi mentor en asignaturas como \textit{Software Process and Quality} y \textit{Desarrollo Avanzado de Software} que me ha enseñado las tecnologías que están en auge en el ámbito del desarrollo web además de ofrecerme los conocimientos básicos y fundamentales de las mismas. Si no huera creído en mí, todos proyectos en los que he participado y de los que tanto he aprendido no hubieran sido posibles.

	\item \textbf{Oscar Peña}, desarrollador del \acrshort{dms} del equipo de MORElab de Deustotech del cual se basa mi proyecto. Gracias a él pude enterarme de las tecnologías que tenía que utilizar para realizar el proyecto y me ayudó a asimilar y comprender el diseño de la base de datos de LabMan para el desarrollo del servidor de \acrshort{oaipmh} así como en el proceso de instalación del servidor de LabMan y a trabajar con Django en general.

\end{itemize}