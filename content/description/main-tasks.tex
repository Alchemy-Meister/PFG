\subsection{Tareas principales}

La implantación del proyecto comprende las siguientes tareas o actividades: 

\subsubsection{Análisis de las herramientas a usar:}

\begin{itemize}
	\item \textbf{Análisis de herramientas provistas por OAI para implementar los requisitos mínimos para repositorio del protocolo OAI-PMH.}

	Investigar las distintas alternativas que hay para crear un servidor que beba de distintos tipos repositorios.
	\item \textbf{Análisis de herramientas para desarrollo web.}

	Investigar las distintas herramientas que hay para el desarrollo web y que sean adecuadas para el propósito del proyecto.
	\item \textbf{Análisis de herramientas semánticas.}

	Investigar las distintas alternativas para realizar búsquedas según los estándares de la web semántica. 
\end{itemize}

\subsubsection{Integración y modelado de datos:}

\begin{itemize}
	\item \textbf{Desarrollo del proveedor.}
	
	Desarrollo del sistema de extracción de datos de las tablas necesarias del repositorio PostgreSQL.

	\begin{enumerate}
		\item Formación: aprendizaje en el uso de las herramientas.
		\item Diseño: diseño del sistema de extracción de datos.
		\item Implementación: programación del sistema de extracción de datos.
		\item Pruebas: pruebas del sistema de extracción de datos.
	\end{enumerate}
	\item \textbf{Diseño del modelo relacional de datos de OAI-PMH.}
	\begin{enumerate}
		\item Diseño: diseño del modelo de la base de datos. 
		\item Implementación: inserción del modelo de datos en la base de datos.
		\item Pruebas: pruebas de la base de datos junto con el sistema de extracción de datos.
	\end{enumerate}
\end{itemize}

\subsubsection{Creación parte servidora del sistema:}

\begin{itemize}
	\item Implementación del servidor OAI-PMH.

	Puesta en marcha del servidor OAI-PMH que transforma datos almacenados mediante un modelo relacional a Dublin Core.

	\begin{enumerate}
		\item Implementación: configuración del servidor.
		\item Pruebas: pruebas del servidor.
	\end{enumerate}
\end{itemize}

\subsubsection{Creación de la aplicación web:}

\begin{itemize}
	\item \textbf{Desarrollo del front-end del sistema.}
	\begin{enumerate}
		\item Formación en la herramienta de desarrollo web.
		\item Diseño básico de la plataforma web.
		\item Diseño del módulo de búsquedas semánticas.
		\item Diseño del módulo de búsquedas facetadas.
	\end{enumerate}	
\end{itemize}

\subsubsection{Validación técnica y de usabilidad:}

\begin{itemize}
	\item \textbf{Pruebas del servidor OAI-PMH.}
	\item \textbf{Pruebas de la plataforma web.}
\end{itemize}

\subsubsection{Documentación y despliegue en producción:}

\begin{itemize}
	\item \textbf{Publicación de la aplicación en el portal web de MoreLab.}
	
	Instalar la aplicación web en el servidor de LabMan y publicarlo en el portal web.
	\item \textbf{Desplegar el servidor OAI-PMH.}

	Instalar el servidor OAI-PMH, recolectar y exportar la información del repositorio.
\end{itemize}