\section{Fortalezas y debilidades de las tecnologías}

Habiendo introducido anteriormente las tecnologías que dispone \acrshort{labman} para su exposición a través del portal web y/o accesible desde las \acrshortpl{api} que proporciona, es necesario realizar un estudio sobre las ventajas y desventajas de la nueva tecnología introducida al sistema respecto a las ya presentes y valorar como afectan al sistema.

\subsection{OAI-PMH}

\subsubsection{Fortalezas}

\begin{itemize}
	\item El protocolo se basa en realizar peticiones \acrshort{http} por lo que no hay que configurar ningún puerto en concreto para hacerlo funcionar.

	\item Al ser un protocolo arraigado en la industria, publicaciones académicas y en las comunidades científicas, convirtiéndose en gigantes ``bibliotecas digitales'', hace que este medio sea perfecto para indexar y explotar grandes volúmenes de información actualizada publicada por estos repositorios.

	\item Las peticiones son sencillas de realizar y no requiere conocimiento alguno de ningún lenguaje de programación. Tan solo de saber estructurar un petición GET o POST con uno de los seis verbos que establece el protocolo \acrshort{oaipmh}, así como definir el formato de los metadatos en los que se desea que se genere la respuesta.
\end{itemize}

\subsubsection{Debilidades}

\begin{itemize}
	\item Al ser un protocolo que gestiona vastas cantidades de información, la gran demanda de los proveedores de servicios podría provocar que las peticiones de estos se conviertan en un ataque de denegación de servicios. Lo que hace que factores como el balanceo de carga, el control de flujo y las redirecciones a servidores secundarios o de ``backup'' sea  esencial por parte de los repositorios. Así mismo es sumamente importante que los clientes no bombardeen a el servidor en caso de encontrarse en situaciones de error y se aconseja que se sigan buenas prácticas a la hora de configurar a los ``robots'' que realizan estas consultas en segundo plano.

	\item El protocolo carece de filtros avanzados para realizar búsquedas. Al ser un protocolo que promueve la preservación de datos digitales, no ha sido diseñado un sistema de búsquedas o filtros avanzados, tan solo permite filtrar por los ``sets'' definidos por el propio repositorio, por fecha inicio y fecha de fin buscando por un identificador en concreto. La responsabilidad de implementar un sistema de búsquedas recae por tanto a el proveedor de servicios y su extracción dependerá del sistema de almacenamiento de la información utilizado por el cliente de \acrshort{oaipmh}.

	\item El rendimiento es dependiente de la implementación del servidor. El proveedor de datos puede recopilar la información de múltiples fuentes, pudiendo optar cualquier tipo de sistema de almacenamiento como ficheros \acrshort{xml}, texto plano, NoSQL\cite{NoSQL}, \acrlongpl{bd} relacionales, entre muchos otros. Esto implica que el rendimiento estará sujeto al tiempo de acceso, filtrado de la información y al tiempo de generación de la respuesta \acrshort{xml} del servidor \acrshort{oaipmh}.
\end{itemize}

\subsection{SQL}

\subsubsection{Fortalezas}

\begin{itemize}
	\item Alto rendimiento, las peticiones \acrshort{sql} pueden ser usadas para acceder a grandes volúmenes de información de la base de datos de manera rápida y eficiente.

	\item El lenguaje ha sido estandarizado por \acrshort{ansi} y por \acrshort{iso}, lo que hace posible que se pueda reutilizar el código \acrshort{sql} en distintas bases de datos teniendo que realizar modificaciones mínimas.

	\item Permite el almacenamiento de objetos en la base de datos, dado que los \acrshort{dbms} orientados a objetos\cite{OODB} son una extensión de las bases de datos relacionales.

	\item Es capaz de realizar consultas complejas, buscando por cada campo definido en las tablas, realizar filtros avanzados con \textit{inner queries}, ejecutar todo tipo de funciones y \textit{joins} con otras tablas, lo que lo convierte en una herramienta idónea para la recolección de datos.
\end{itemize}

\subsubsection{Debilidades}

\begin{itemize}
	\item Aunque las \acrlongpl{bd} \acrshort{sql} se conforman a los estándares \acrshort{ansi} e \acrshort{iso} muchas de las \acrlongpl{bd} \acrshort{sql} implementan algunas de sus funcionalidades como extensiones propietarias para asegurar la permanencia de sus usuarios.

	\item Realizar consultas a una tabla es meramente intuitivo, pero a medida que se añaden condiciones, filtros, funciones, \textit{inner queries} y \textit{joins} con otras tablas la dificultad aumenta, haciendo el lenguaje difícil, para consultas avanzadas.
\end{itemize}

\subsection{SPARQL}

\subsubsection{Fortalezas}

\begin{itemize}
	\item Alto soporte a consultas semiestructuradas, como por ejemplo a datos con estructuras impredecibles o poco fiables.

	\item \acrshort{sparql} realiza consultas a grafos \acrshortpl{rdf} y estos están compuestos por varias tripletas que expresan relaciones binarias entre los recursos. Esta característica permite a las consultas \acrshort{sparql} realizar \textit{joins} implícitas entre los distintos recursos, convirtiendolo en un lenguaje apropiado para realizar búsquedas en fuentes de información dispares en una sola consulta.

	\item \acrshort{sparql} ha sido diseñado para soportar consultas en un entorno web, en los que los nombres de los grafos se identifican mediante \acrshortpl{uri}, por otra parte es común que las implementaciones de \acrshort{sparql} recojan la información de los grafos por medio peticiones \acrshort{http} GET sobre las \acrshortpl{uri} de los grafos.

	\item Existen herramientas que permiten el uso de \acrshort{sparql} para consultar contenido que no está almacenado en \acrshort{rdf}, tales como \acrshort{ldap}, \acrshort{xml}, \acrshort{sql}, entre otros. Entre estas herramientas caben destacar el servidor D2R\cite{D2R_Server}, el Adaptador Jena para bases de datos Oracle\cite{JENA_ORACLE}, SquirrelRDF\cite{SquirrelRDF}, etc.\cite{MAPING_SPARQL} que ofrecen servicios de mapeo de \acrfull{s2s}.

	\item Así mismo la palabra clave de un grafo permite adquirir datos del lugar de origen de dicha información. Los grafos pueden ser usados para descubrir las \acrshortpl{uri} de grafos que contengan los datos que coincidan con dicha consulta en cuestión.
\end{itemize}

\subsubsection{Debilidades}

\begin{itemize}
	\item Por lo general, las consultas en \acrshort{sparql}, de no usar estrategias el cacheo de consultas\cite{SPARQL_Performance}, son muy lentas.

	\item Al ser una tecnología joven, no son muchos los que hacen uso de esa tecnología, es por eso que aún no haya muchos repositorios de datos que puedan recolectarse por medio de consultas \acrshort{sparql} comparado al basto despliegue de otras tecnologías veteranas como \acrshort{sql} o \acrshort{xpath}\cite{XPath}.

	\item No da soporte a las consultas jerárquicas o transitivas. \acrshort{sparql} no facilita realizar consultas que presenten relaciones transitivas o estructuras jerárquicas dentro de un grafo, algo que por el contrario si soporta \acrshort{xquery} mediante los \textit{Axis}, capaz de relacionar un nodo en cuestión, con su padre y sus hijos\cite{XQueryAxes}.

	\item Es una tecnología inmadura en la que se hace notar la ausencia de procesamiento explícito como el de \acrshort{xquery} o la optimización del ya veterano \acrshort{sql}.
\end{itemize}

\subsection{Conclusiones}

\acrshort{sql} a es una tecnología extendida y asentada que ha demostrado ser eficiente capaz de gestionar grandes volúmenes de información en consultas centralizadas. Por otra parte \acrshort{oaipmh} además de fomentar el acceso a las publicaciones digitales, sirve como complemento a esta tecnología permitiendo recolectar información desde distintas fuentes para su adición a una \acrshort{bd}.

Sin embargo, \acrshort{sparql}, va más allá permitiendo gestionar la información en múltiples \acrlongpl{bd} distribuidas en una única consulta, evitado la redundancia de datos, los potenciales quebraderos de cabeza de realizar \textit{joins} consecutivas y la necesidad de tener que disponer de toda la información en un único sistema.

Es por ello que debido al menor rendimiento de \acrshort{sparql} respecto a \acrshort{sql}, \acrshort{labman} prioriza el rendimiento de la plataforma web haciendo uso de una base de datos relacional PostgreSQL\cite{PostgreSQL} para gestionar la información de las publicaciones y los proyectos. Por otra parte, cabe destacar que cada vez que se guarda una instancia en el modelo relacional, por medio de un \textit{trigger} se genera las tripletas de ese mismo recurso y se almacenan en un \textit{triple-store} Virtuoso\cite{Virtuoso}. El objetivo de este servidor es generar una copia semántica de la información almacenada en la \acrshort{bd} relacional, permitiendo así preparar una futura migración  de los ya \textit{legacy} \acrfullpl{dbms} relacionales a el que se cree que es la tecnología del futuro para la gestión de la información, las \acrshortpl{bd} semánticas. Además, este servidor permite realizar consultas complejas \acrshort{sparql} beneficiosos para la generación los gráficos sobre la información financiera de los proyectos en los que toma parte el grupo de investigación.

 Como conclusión, además de facilitar la transición tecnológica a las \acrshortpl{bd} semánticas, el verdadero objetivo de este proyecto es poder facilitar la información para ser explotada de la forma más eficiente posible. Es por eso que se ve la necesidad de implantar una \acrfull{api} \acrshort{oaipmh} para que los muchos sistemas que aún hoy en día hacen uso de esta tecnología para la recolección de datos de y su posterior gestión centralizada.