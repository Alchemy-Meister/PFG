\section{Introducción a LabMan}

Es este apartado se realiza una breve introducción a \acrfull{labman}, el \acrlong{dms} de MoreLab, el grupo de investigación formado por los equipos de Internet y Telecomunicaciones de DeustoTech y en el que se basa este proyecto.

\begin{figure}[!htp]
	\centering
	\includegraphics[scale=0.15]{fig/morelab-logo}
	\caption{Logo de MORELab}
\end{figure}

Gestionar la información no siempre es una tarea trivial y más aún cuando hay que tratar con los datos que componen varias entidades como pueden ser los proyectos, investigadores, publicaciones, eventos, etc. dentro del ámbito de la investigación. La mayoría de los grupos de investigación utilizan sistemas de gestión de contenido tales como Joomla!\cite{joomla}, WordPress\cite{wordpress} o Drupal\cite{drupal} para exponer sus datos. Sin embargo, para extraer la información de estos \acrshortpl{cms} se requieren herramientas externas para llevar a cabo técnicas de análisis de datos. 
Para hacer uso de estas herramientas normalmente hace falta generar documentos adicionales, tales como \acrshort{csv}, ficheros de texto, entre otros que provocan redundancia de la información, que provoca dificultades a la hora de actualizar los datos y la calidad de los mismos. Esta situación empeora cuando además se disponen de distintas fuentes para la obtención de información, como pueden ser las paginas web personales de los investigadores en los que se muestran sus logros a lo largo de su carrera, la información financiera gestionada por su propio departamento, etc.

Del esfuerzo para gestionar la información grupo de investigación de MoreLab nace {labman}, con el objetivo de gestionar todo este tipo información diferenciadose de otros \acrshort{cms} por apostar por la exposición de los datos como \acrlong{lod}\cite{linkeddata}.

Los datos enlazados\cite{pena_visual_2014}