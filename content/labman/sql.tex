\section{Introducción a SQL}

Como su propio nombre indica, \acrfull{sql} es un lenguaje de programación, estandarizado por \acrshort{iso}\cite{ISO} y \acrshort{ansi}\cite{ANSI}, diseñado para gestionar la información almacenada en los sistemas de gestión bases de datos relacionales, como por ejemplo PostgreSQL\cite{PostgreSQL}, MySQL\cite{MySQL}, SQLite\cite{SQLite}, por medio de consultas estructuradas en inglés.

Las consultas representan las operaciones más comunes y esenciales del lenguaje \acrshort{sql} para la recopilación de información dentro de una \acrshort{bd}. Estas consultas se realizan por medio de la sentencia ``SELECT'', que pueden ser complementadas por medio de clausulas para realizar búsquedas específicas. De todas las clausulas disponibles hay que destacar la siguientes:

\begin{itemize}
	\item \textbf{FROM:} Indica de que tabla o tablas de debe extraerse la información.
	\item \textbf{WHERE:} Sirve para delimitar las tuplas o filas de la tablas en las que se realiza la búsqueda. Si las filas no cumplen las condiciones expuestas en esta clausula, serán excluidas del resultado.
	\item \textbf{ORDER BY:} Esta es la única forma para definir el criterio de ordenación los resultados en \acrshort{sql}, sin esta clausula el orden sería aleatorio.
\end{itemize}

\lstinputlisting[language=SQL, frame=single, label={lst:selectsql}, caption=Ejemplo de sentencia SELECT en \acrshort{sql}]{content/code/sql/select-example.sql}

En esta consulta \acrshort{sql} (ver algoritmo \ref{lst:selectsql}) se solicitan todos los nombres y las ciudades de los clientes que residan en Suecia.