\chapter{Tecnologías utilizadas}

En esta sección se presentan las diferentes tecnologías que se han utilizado para el desarrollo del proyecto. Las principales tecnologías usado son las siguientes:

\section{Servidor MOAI}

MOAI\cite{MOAI} es una plataforma capaz de recolectar información de diversas fuentes y publicarlas mediante el protocolo \acrshort{oaipmh} desarrollado por Infrae\cite{Infrae} con el fin de satisfacer las necesidades de las instituciones académicas que trabajan con metadatos relacionales y ficheros de información.

Basado en un servidor \acrshort{http} Apache\cite{HTTPApache} e implementado en Python\cite{Python}, MOAI incluye todas las funciones básicas como proveedor de datos del protocolo \acrshort{oaipmh}, indispensables para este proyecto. Por otra parte, permite tanto la extracción la información de documentos \acrshort{xml} como recolectarla de diversos servidores de información tales como Fedora Commons\cite{Fedora}, EPrints\cite{EPrints} o DSpace\cite{DSpace} y almacenar la información cosechada en una base de datos SQLite\cite{SQLite} por defecto.

\subsection{Razón de uso}

Se ha escogido MOAI Server frete a las alternativas facilitadas por la misma comunidad de \acrshort{oai} en \url{https://www.openarchives.org/pmh/tools/tools.php} entre las que se cabe destacar a Fedora, EPrints y DSpace por las siguientes razones:

\begin{itemize}
	\item Su accesible documentación para el desarrollo y extensión del servidor.
	\item Capacidad para sobrescribir el esquema de la base de datos SQLite usado para la extracción de datos del servidor Apache por defecto y sustituirlo por la \acrshort{bd} PostgreSQL\cite{PostgreSQL} utilizada actualmente por LabMan.
	\item Implementado en una tecnología conocida, lo que reduce el tiempo de adaptación a la capa de datos.
\end{itemize}

\section{SQLAlchemy}

\section{Django}

\section{JQuery}

\section{Bootstrap-select}

\section{Bootstrap 3 Datepicker}

\chapter{Presupuesto}

En esta sección se detalla el presupuesto planificado para la realización de proyecto. El presupuesto se desglosa en tres apartados diferentes: Recursos Humanos, Software y Hardware.

\subsection{Recursos Humanos}

En la tabla \ref{tab:budget-human} se detalla los honorarios por cada uno de los roles, dependiendo de las horas que trabaja según lo planificado.

\begin{table}[htp]
	\centering
	\caption{Presupuesto de Recursos Humanos}\label{tab:budget-human}
	\begin{tabular}{cccc}
		\toprule
    	\textbf{Rol} & \emph{Precio/hora(\euro/h)} & \emph{Carga de trabajo(h)} & \emph{Importe total(\euro)}\\
    	\midrule
    	Jefe de proyecto				&	40			&	6,05 					& 	242,00\\
		Administrador de base de datos	&	25			&	64,05					&	1.601,25\\
		Programador						&	25			&	185,48					&	4.637,00\\
		Diseñador						&	15			&	26,22					&	393,30\\
		Experto en web semántica		&	30			&	26,22					&	786,60\\
    	\bottomrule
    \end{tabular}
\end{table}
\subsection{Recursos software}

En la tabla \ref{tab:budget-software} se detalla el gasto destinado a las herramientas software según lo planificado.

\begin{table}[htp]
	\centering
	\caption{Presupuesto de software}\label{tab:budget-software}
	\begin{tabular}{cccc}
		\toprule
    	\textbf{Nombre} & \emph{Precio(\euro)} & \emph{Unidades} & \emph{Importe total(\euro)}\\
    	\midrule
    	Licencia Sublime Text 2		& 	70				&	1 			& 	70\\
    	Office 2011 				&	99				&	1			&	99\\
    	\bottomrule
    \end{tabular}
\end{table}
\section{Recursos Hardware}

\begin{table}[htp]
	\centering
	\caption{Presupuesto: Hardware}\label{tab:budget-hardware}
	\begin{tabular}{cccc}
		\toprule
    	\textbf{Nombre} & \emph{Precio(\euro)} & \emph{Unidades} & \emph{Importe total(\euro)}\\
    	\midrule
    	MBPR2012			& 	3.334			&	1 			&	3.334					\\
    	Monitor secundario	&	300				&	1			&	300						\\
    	\bottomrule
    \end{tabular}
\end{table}

\clearpage

\subsection{Total}

Para finalizar, en la tabla \ref{tab:total-budget} se detalla el presupuesto total, que se calcula mediante la suma del costo estimado de los Recursos Humanos, el Software y el Hardware.

\begin{table}[htp]
	\centering
	\caption{Presupuesto total}\label{tab:total-budget}
	\begin{tabular}{cc}
		\toprule
    	\textbf{Tipo} 		& 	\emph{Total}\\
    	\midrule
    	Recursos Humanos	& 			7.660,15\\
		Recursos Software	&			169,00\\
		Recursos Hardware	&			3.634,00\\
		\textbf{Total}		&	\textbf{11.463,15}\\
    	\bottomrule
    \end{tabular}
\end{table}