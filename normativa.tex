\section{Introducci\'on}

En este documento se recogen las especificaciones más relevantes relacionadas con el formato de
las memorias de los proyectos de la asignatura \textbf{Proyecto Fin de Grado} del Grado en Ingeniería
Informática, Grado en Ingeniería en Tecnologías Industriales, Grado en Ingeniería en Electrónica
Industrial y Automática, Grado en Ingeniería en Organización Industrial y Grado en Ingeniería
en Tecnologías de Telecomunicación.

\section{Encuadernaci\'on}

La encuadernación será en cartulina (con cola y cinta negra), del color especificado a continuación
según titulación (las tonalidades específicas están disponibles en la cartelera).

\begin{center}
  \begin{tabular}{ll}
    \toprule
      \textbf{Titulaci\'on} & \textbf{Color de la encuadernación}\\
    \midrule
      Grado en Ingeniería Informática                             &  Azul mediterráneo \\
      Grado en Ingeniería en Organización Industrial              &  Crema             \\
      Grado en Ingeniería en Electrónica Industrial y Automática  &  Rojo Navidad      \\
      Grado en Ingeniería en Tecnologías de Telecomunicación      &  Verde Hierba      \\
      Grado en Ingeniería en Tecnologías Industriales             &  Butano            \\
    \bottomrule
  \end{tabular}
\end{center}

\section{Portada}

La portada consta de un encabezado que respeta la identidad corporativa de la Universidad de
Deusto y que se facilita a los estudiantes a través de la página web de la asignatura. En dicho
encabezado figuran en castellano y en euskera los datos relativos a la Universidad, la Facultad y la
titulación correspondiente.

\textbf{Dada la complejidad de definir literalmente toda la distribución de los diversos contenidos de la
portada, se facilitan plantillas en formato \enquote{pdf editable} por cada titulación y \emph{que son de
obligado uso}.}

El orden de los párrafos será:
\begin{itemize}
  \item La expresión: \textbf{Proyecto fin de grado}.
  \item El título completo (sólo se escribirán con mayúscula la primera letra de la primera palabra
        y los nombres propios; no se pone punto al final)
  \item El nombre del estudiante
  \item El director o directora, donde se especificará dicho cargo, tal como figura en el ejemplo, es
        decir: \textbf{Director}: o \textbf{Directora}:. No se debe especificar el título de éste (doctor, licenciado o
        ingeniero)
  \item  La fecha, tal como figura en el ejemplo. \textbf{Bilbao, mes de año} (en número)
\end{itemize}

\section{Papel}

Ha de emplearse papel del mismo color (blanco), calidad 80 gr. mínimo y normalmente suficiente y
tamaño (UNE A-4), en todas las copias de la memoria.

\section{Impresi\'on}

\textbf{Se empleará la plantilla \enquote{PlantillaPFG}} disponible en la página web de la asignatura (en la sección
de Plantillas) con el fin de respetar las fuentes, los márgenes y los encabezados y pies de páginas.
Se suministra un ejemplo de su uso en formato Word y \hologo{LaTeX}.

\textbf{El uso de estas plantillas (y no otras) es de obligado cumplimiento} para lograr una uniformidad en
todas las titulaciones y memorias. Se deberán mantener el formato de las fuentes, títulos y
márgenes establecidos en estos documentos.

Los márgenes han de ajustarse a las siguientes medidas: superior e inferior 3 cm, lateral interno 3’5
cm y lateral externo 2’5 cm.

El encabezado de las \textbf{páginas pares indicará el título del capítulo en curso} y el encabezado de las
\textbf{páginas impares llevará los términos \enquote{PROYECTO FIN DE GRADO}}, según plantilla. La numeración
de las páginas será exterior. Las páginas en blanco no deben llevar numeración ni encabezado y las
páginas posteriores deben seguir la secuencia de la numeración del documento.

Si hay alguna imagen, tabla o gráfico que deban ir en horizontal, se debe rotar la imagen, no la
página completa. El encabezado y pie de página debe ser los mismos en toda la memoria, a
excepción de los anexos, que pueden llevar un formato distinto, pero debe estar indicado
claramente que se trata de anexos.

La presentación de la memoria debe cuidarse con especial esmero, procurando claridad, limpieza,
uniformidad y ausencia de erratas. Se imprimirá a doble cara. Se evitarán los saltos de página
innecesarios y la presencia de páginas que por una u otra razón estén prácticamente en blanco.

Las tres copias de la memoria que se entregan deben ser idénticas y la impresión se puede realizar
en B/N o color.

\section{Organización del contenido}
\subsection{Primera página}
Debe ser exactamente igual que la portada, pero en papel blanco de la misma calidad que el
resto de la memoria.

Esta página es la que \textbf{deberá llevar a modo de confirmación la firma del director/a del
proyecto. Esta firma deberá estar encima del nombre del director/a.}

\subsection{Segunda página}
La segunda página es el reverso de la primera página, irá en blanco (sin número ni encabezado).

\subsection{Resumen y descriptores}
En la tercera página (anverso de la segunda hoja) debe aparecer un resumen del proyecto (de
200 a 250 palabras) y a continuación del mismo, entre tres y cinco descriptores (palabras clave)
que ayuden a clasificar adecuadamente el proyecto.

La paginación anterior al capítulo 1 (portada, resumen e índice) se hará con números romanos
(i, ii, iii, iv,\dots). En la primera página, la portada, no debe aparecer el número.

\subsection{\'Indice}
El índice empezará en la página 5 (v), es decir, en el anverso de la tercera hoja.

\subsection{Memoria}
El capítulo 1 comenzará en la página 1. El resto de capítulos \textbf{deberán comenzar también en
página impar}.

\subsection{Bibliograf\'ia}

Este será el último capítulo del documento e irá antes de los anexos. El formato de la
bibliografía es el indicado en los ejemplos de memoria y plantillas disponibles, tanto en \hologo{LaTeX}
como en Word. De modo general, el formato a emplear será:

\begin{description}
  \item [Para referenciar páginas web:] \enquote{Titulo/Nombre de la Web},
        \url{http://www.google.es}, (consultado el 7/10/12).
  \item [Para referenciar libros:] Autores separados por comas, \enquote{Título de la contribución/Libro},
        \emph{Editorial en cursiva}, Año de la publicación.
\end{description}

\subsection{Planos}
Si los hubiera, la presentación de planos se hará de acuerdo a la normativa vigente (UNE-EN ISO
7200:2004) en esta área.

\subsection{Anexos}
Los anexos irán adjuntados al final de la memoria y podrán tener un formato libre diferente al
del resto de la memoria.

\subsection{Otros}
No se establece ninguna otra restricción en cuanto al contenido y/o formato de las memorias, a
excepción de las que se recogen en el presente documento.

